\section{Methodology}
\label{sec:methodology}

In this section, we introduce key steps of our works. We first detail how the architecture of SketchPointNet is designed. Next, we will show how to resample each sketch into a fixed number of points. At last, we explain that our group scheme guarantees the integrity of group patterns.

\subsection{SketchPointNet}
\label{ssec:sketch_point_net}

SketchPointNet recognizes sketches as time series. First, we resample each sketch into $N$ points. Second, in order to get local pattern information of each sketch, we group sketches for two times. Micro PointNet P1 summarizes patterns with group radius $r_1 = 0.1$. Micro PointNet P2 summarizes patterns in a large receptive field with $r_2 = 0.3$. At last, PointNet P3 summarizes pattern features generated from P2 for classification of 250 categories.

In detail, given a list of strokes $T^{init}$, we resample each sketch into N points. Each point has 3 dimension. The first two dimension denote the location of the point. The last dimension is the stroke order information(each point belong to a stroke). In order to extract features of local patterns, we resample $N_{g1}$ points as group centers every $t_1$ points and group the points around group centers with $r_1 = 0.1$ from the $N$ points. After the first time of grouping, we get $N_{g1}$ groups $\left\{ g_{1_i}| i = 1, ..., N_{g1} \right\}$. Each group contains $S_{g1}$ points. Next, we calculate relative position of group points according to group centers in each group. Then we get a $D_{g1}$-d pattern feature after each of $N_{g1}$ groups passing through the micro PointNet P1. Now, each group center corresponds to its group pattern feature.

In a similar way, we resample points every $t_2$ points from the last group centers. Then we get $N_{g2}$ larger groups $\left\{ g_{2_i}| i = 1, ..., N_{g2} \right\}$ after grouping $g_1$ with $r_2 = 0.3$. Each larger group contains $S_{g2}$ smaller groups from $g_1$. We get $D_{g2}$-d feature as $g_2$ pattern features after feeding $S_{g2}$ smaller groups into micro PointNet P2. Concating $g_2$ pattern features and their corresponding centers($N_{g2} \times (D_{g2}+3)$). we feed them into PointNet P3. At last, SketchPointNet output 250 scores corresponding to 250 categories.

SketchPointNet architecture(see Fig.~\ref{fig:sketchpointnet}) are borrowed from PointNet++ architecture. The differences between SketchPointNet and PointNet++ are how we select group centers and group the points. For PointNet++, they use iterative farthest point sampling(FPS) to choose group centers to ensure that group centers are evenly distributed. Sketch points are sparsely distributed on the strokes. We choose group centers every several points along the time series(each sketch represented as a time series). And sketches are different from 3d point clouds. For sketches, points densely distribute in the places where locate the patterns. FPS sample no more points in the place, which will make patterns lost. Besides, the way we choose group centers are more efficient than the way in PointNet++. When grouping the points around group centers, we have to set a proper number of points for each group. In PointNet++, if a number of group points  is greater than the setting number, they directly delete last extra points in the group. But we delete points every other point till the number of points in groups equals to the setting number. This can ensure the integrity of patterns(see Fig. ~\ref{fig:group}).

\subsection{Resample scenario}
\label{ssec:resample_scenario}

For each sketch, we sample it into a fixed number of points. Or we can sample each points in sketches with a given interval. In second case, the point number $\left\{ N_{P_i}| i = 1, ..., 20000 \right\}$ of sketches are different due to various total length of strokes of each sketch. The input to the network must have a fixed size. Because tensorflow 1.2 only support static graph. If we set input size as $\max \limits_{1 \le i \le 20000} N_{P_i}$, there are great waste for computing resources. Because only few sketches need large number of points to represent.

\begin{figure}
    \center
    \includegraphics[width=3in]{images/resample2.png}
    \fcaption{Resample the airplane and the tree into $N$(256 and 128) points.}
    \label{fig:resample}
\end{figure}

 We calculate each stroke length $\left\{ L_i| i = 1, ..., m \right\}$ from strokes $\left\{ T_i^{init}| i = 1, ..., m \right\}$. $m$ stands for stroke numbers. The total stroke length is $C$. Given the number of points $N$, we assign each stroke $B_i$ points $\left\{ B_i| i = 1, ..., m \right\}$ according to its corresponding stroke length $T_i$. Then we reample each stroke into $B_i$ points by using \$1 \cite{Wobbrock2007GesturesWL} resample scheme. The strokes after resampled are represented as $\left\{ T_i^{rsp}| i = 1, ..., m \right\}$. The total number of points in strokes $T^{rsp}$ is $N$. So, each sketch is represented as a $N$ points time series. In Fig.~\ref{fig:resample}, the airplane and the tree are resample into different number of points.

\begin{algorithm}
\label{alg:resample}
    \caption{Resample each sketch into N points}
    \KwIn{$\left\{ T_i^{init}| i = 1, ..., m \right\}$, resample number $N$}
    \KwOut{$\left\{ T_i^{rsp}| i = 1, ..., m \right\}$}
    $C = 0$\;
    \For{$ i = 1; i \le m $}
    {
        $L_i = |T_i^{init}|$\;
        $C = C + L_i$\;
    }

    \For{$ i = 1; i \le m $}
    {
        $B_i = \frac{L_i \times N}{C}$\;
        $T_i^{rsp} = single\_stroke\_resample(T_i^{init}, B_i)$\;
    }
    return $\left\{ T_i^{rsp}| i = 1, ..., m \right\}$\;
\end{algorithm}



\subsection{Group scheme}
\label{ssec:group_scheme}

After resampling sketches, we get each of them represented as N points $\left\{P_i| i = 1,..., N\right\}$. For a sketch, We resample a series of points $\left\{P_{1}, P_{1+t}, P_{1+2*t}, ..., P_{1+(N_g-1)*t}\right\}$ every $t$ points as group center from the N Points. Then we group points according to group center within radius $r$. After grouping the sketch, we get $N_{g} \times S_{g}^{init}$. $N_{g}$ is the number of the groups. $S_{g}^{init}$ is the points number of each group. In a similar way, we group the sketch for more times according to last grouping center.

The input group must have a fixed point number $S_g$. But with a fixed group radius $r$, group point numbers are very different from each other. After grouping the points, We get a series of groups. One of the groups contains $S_g^{init}$ group points $\left\{ Z_i| i = 1, ..., S_g^{init} \right\}$. If the point number $S_g^{init}$ is less than $S_{g}$, we pad the group center points into the group till the group size is $S_{g}$. If the point number $S_g^{init}$ is greater than $S_{g}$, we delete group points every other points in the group till the point numbers is $S_{g}$. We develop an algorithm to guarantee the integrity of group patterns when deleting extra points.

\begin{algorithm}
\label{alg:group}
    \caption{Delete extra points until group size is $S_g$}
    \KwIn{$\left\{ Z_i^{init}| i = 1, ..., m \right\}$, group size threshold $S_g$}
    \KwOut{$\left\{ Z_i| i = 1, ..., S_g \right\}$}
    $i = 2$\;
    $Z = Z^{init}$\;
    \While{$|Z| < S_g$}
    {
        \If{$i = |Z|+1$ or $i = |Z|+2$}
        {
            $i = 2$\;
        }
        del $Z_i$\;
        $i = i+1$\;
    }
    return $\left\{ Z_i| i = 1, ..., S_g \right\}$\;
\end{algorithm}

\begin{figure}
    \center
    \includegraphics[width=3in]{images/group.png}
    \fcaption{Comparison of dealing with extra points.}
    \label{fig:group}
\end{figure}

All N points in a sketch are chronological. Grouping do not undermine the order of time series. The points in a group are still arranged in original chronological order. In Fig.~\ref{fig:group} (B), The initial group contains $S_g^{init}$ points which is greater than $S_g$ points. If we directly delete the last $|S_g^{init}-S_g|$ points. From Fig.~\ref{fig:group} (C), we can see that there are great loss of the pattern.  Algorithm ~\ref{alg:group} makes sure of the existence of the whole pattern(see Fig.~\ref{fig:group} (D)).


