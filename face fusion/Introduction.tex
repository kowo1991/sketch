\section{Introduction}



\cxj{TBD: title: Fusing faces by what? or based on? }


<<<<<<< HEAD
These years, there has been a great need for face editing application.
Someone would wonder what they look like if they have some face features from pop stars, and they may make their faces sharper through face editing techniques.
Some other people want to know what they look like when they grow older.
These tasks could be accomplished by face fusion.
Progeny appearance prediction is a popular task lately. What the son of a young couple looks like when he grows up. It is attracting for many people, and this could also be accomplished by face fusion technology.

Face morphing creates correspondences by meshing face images\cite{mhf,wol}. Thus, face can be warped according to the changes on  the mesh. But the intermediate results of face morphing are simply interpolated by two warped face images. So it might be unnatural. What we want is to create an natural face image from two faces. Face swapping can stitch one's face to another. Their main task is dealing with pose variation. Ours are not. We only use frontal face. So we can create interpolated structures of two faces. Both face morphing and face swapping don't consider hues and lightning conditions of two faces.
=======
These years, there has been a great need for face editing application. 
Someone would wonder what they look like if they have some face features from pop stars, and they may make their faces sharper through face editing techniques. 
Some other people want to know what they look like when they grow older. 
These tasks could be accomplished by face fusion. 
Progeny appearance prediction is a popular task lately. What the son of a young couple looks like when he grows up. It is attracting for many people, and this could also be accomplished by face fusion technology.

\cxj{A paragraph introducing traditional face fusion/editing techniques. Address the problem and challenges. }

>>>>>>> origin/master

As we know, different faces have different features. Someone has a sharper face while the shapes of other faces are closer to square. There are many ways to build correspondences of two faces. In early years, hand-marked features of faces are used to build correspondences of two faces \cite{fbim}. Because of the lack of the technology to detect face features automatically, when we align two faces using this method, there are a lot of labour work to do. Lately, a five feature points based face morphing method \cite{mhf} was proposed. The five feature points are detected automatically. Then with these feature points, two faces could be warped and fused into one. In fact, five feature points are not enough to represent a face. When we use the sparse correspondences to generate new faces, the results are weird.

Given two faces, source face is the face that we want to use its features to influence other faces, while target face is the face waited to influenced. There are 68 facial landmarks detected by using Ensemble of Regression Trees \cite{fld} in our method. Thus we estimate the distributions of the features and the size of the face. Facial landmarks indicate face structures, while face gradients indicate face details. When we fuse two faces, first, we warp two faces to make sure the structures of two warped faces are the same. Here, we propose a method  to create a new set of landmarks by interpolating two groups of given landmarks. Then, we fuse the gradients of two faces. In order to maintain the hues and illumination information of the target face, we reconstruct the gradients for new faces by preserving the weaker gradients of the target face and fusing the stronger gradients of the two faces. Then, we recover the fused face by seamless clone \cite{pie}.

Our contributions can be summarized as follows. First, we propose a new method to fuse features of source face into target face. Second, we develop an algorithm to reconstruct gradients of fused face which keeps lots of facial hues and illumination information of target face, so the fused face looks more photorealistic. Third, we develop a user interface, for which users can decide how many features we want to fuse into target face from source face.
